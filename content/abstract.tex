%!TEX root = ../main.tex

\pagestyle{empty}

% override abstract headline
\renewcommand{\abstractname}{Abstract}

\begin{abstract}

    Die Arbeit untersucht den Einsatz von Large Language Models (LLMs) zur Vorauswahl von Bewerbern in der IT-Branche. Ziel ist es, durch den Einsatz eines LLM-gestützten Chatbots, der simulierte Bewerbungsgespräche führt, den Rekrutierungsprozess zu optimieren. Das Experiment bewertet Bewerber anhand von vier Kriterien: fachliche Kompetenz, Problemlösungsfähigkeit, Konfliktlösungsfähigkeit und Teamfähigkeit. Die Ergebnisse zeigen, dass das LLM Potenzial zur Objektivierung und Effizienzsteigerung im Auswahlprozess besitzt, jedoch durch Tendenzen zu positiven Bewertungen und Schwierigkeiten bei der Bewertung von Soft Skills beeinträchtigt wird. Anpassungen der Bewertungsmechanismen und eine stärkere Gewichtung fachlicher Kompetenzen sind erforderlich, um eine präzisere Differenzierung der Bewerber zu gewährleisten. Die Forschung liefert wertvolle Erkenntnisse zur Automatisierung von Rekrutierungsprozessen, zeigt jedoch auch die Notwendigkeit menschlicher Überwachung und kontinuierlicher Verbesserung auf.

\end{abstract}