%!TEX root = ../../main.tex

\chapter{Einleitung}

\section{Motivation}

In einer zunehmend digitalisierten Welt wird der Einsatz von Künstlicher Intelligenz (\acs{KI}) immer wichtiger, insbesondere in personalintensiven Bereichen wie dem Recruiting. 
Die Personalauswahl ist dabei ein zeitaufwendiger und oft subjektiver Prozess, der in der IT-Branche durch die spezifischen Anforderungen an technisches Fachwissen weiter erschwert wird. 
Unternehmen stehen vor der Herausforderung, aus einer Vielzahl von Bewerbern schnell und effizient die besten Kandidaten zu identifizieren. 
Der Einsatz von \acs{KI}-Tools bietet eine Möglichkeit, diesen Prozess zu optimieren, indem er die Bewerbervorauswahl automatisiert und objektiviert. 
Mit Hilfe von \acp{LLM} können Screening-Prozesse durchgeführt werden, die sowohl fachliche als auch soziale Kompetenzen der Bewerber analysieren und bewerten. 
Dies reduziert nicht nur den Zeitaufwand, sondern minimiert auch Verzerrungen im Auswahlprozess.

\section{Zielsetzung}

Ziel dieser Arbeit ist es, die Entwicklung und den Einsatz eines \ac{LLM}-gestützten Chatbots zur Vorauswahl von Bewerbern im IT-Sektor wissenschaftlich zu untersuchen. 
Der Chatbot soll in der Lage sein, simulierte Bewerbungsgespräche zu führen, die Bewerber auf Basis ihrer Antworten zu bewerten und eine Entscheidungshilfe für die endgültige Auswahl zu bieten. 
Dabei soll analysiert werden, inwiefern das System die Bewerberkompetenzen präzise erfasst, wie es zu einer objektiveren Bewerberbewertung beitragen kann und welche Anpassungen notwendig sind, um eine fehlerfreie Differenzierung zwischen geeigneten und ungeeigneten Kandidaten zu gewährleisten.
