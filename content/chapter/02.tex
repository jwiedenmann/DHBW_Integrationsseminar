%!TEX root = ../../main.tex

\chapter{Stand der Technik}
Der Rekrutierungsprozess ist traditionell in mehrere Schritte unterteilt. Dazu gehören laut Nikolaou die Anziehung geeigneter Kandidaten, deren Bewertung und Auswahl sowie die Integration der neuen Mitarbeiter. Dieser Prozess ist in jedem Unternehmen sehr zeit- und kostenintensiv \cite{nikolaou:2021}. 

\acs{KI} kann den Rekrutierungsprozess in vielerlei Hinsicht effizienter gestalten. Sie ermöglicht die Automatisierung zeitaufwendiger Aufgaben wie das Screening und die Vorauswahl von Bewerbern. Kong et al. betonen ebenfalls, dass \acs{KI}-basierte Tools das Potenzial haben, die Effizienz von Personalabteilungen erheblich zu steigern \cite{kong:2021}. Durch den Einsatz dieser Technologien können Unternehmen einen größeren Talentpool verwalten und skalierbare Rekrutierungsprozesse aufbauen, was besonders bei einer großen Anzahl von Bewerbern vorteilhaft ist.

Der Einsatz von \acs{KI} im Rekrutingprozess ist jedoch auch mit Bedenken verbunden. Ein zentrales Thema beim Einsatz von \acs{KI} im Rekrutierungsprozess ist das Risiko der Voreingenommenheit (Bias). Während \acs{KI}-Systeme als objektive Tools wahrgenommen werden, können sie dennoch bestehende Ungleichheiten und Diskriminierungen abbilden oder sogar verstärken. Drage und Mackereth hinterfragen, ob \acs{KI} fähig ist, Geschlecht und Ethnie vollständig unberücksichtigt zu lassen. Sie argumentieren, dass diese Merkmale oft falsch als isolierbar angesehen werden, während sie in der Realität tief in sozialen und historischen Strukturen von Macht und Einfluss verwurzelt sind \cite{drage:2022}. Als bekanntes Beispiel wird von Hunkenschroer und Luetge das \acs{KI}-gestützte Rekrutierungstool von Amazon angeführt, welches männliche Kandidaten bevorzugte, da es auf historischen Daten trainiert wurde, welche überwiegend von männlichen Top-Performern stammten. Die Autoren weisen zudem darauf hin, dass selbst gut entwickelte Systeme anfällig für indirekte Diskriminierung sind, zum Beispiel durch die Verwendung von Proxy-Variablen, die auf sozioökonomische Faktoren hindeuten \cite{hunkenschroer:2022}.

Ein weiterer kritischer Aspekt ist die Bewertung von Soft Skills durch \acs{KI}. Fähigkeiten  wie Kommunikationsstärke, Teamfähigkeit und emotionale Intelligenz sind in vielen Positionen relevant, lassen sich jedoch nur schwer durch automatisierte Prozesse bewerten. Rigotti und Fosch-Villaronga betonen, dass \acs{KI}-Systeme in Bereichen wie Intuition, Emotion und kontextueller Sensibilität den Menschen unterlegen sind. Diese menschlichen Eigenschaften sind jedoch von entscheidender Bedeutung für fundierte Entscheidungen im Rekrutierungsprozess. Daher bleibt menschliches Eingreifen unerlässlich, um sicherzustellen, dass die Technologie ethisch und effektiv eingesetzt wird \cite{rigotti:2024}. Alami et al. schlagen daher vor, bei der Vorauswahl von Bewerbern, insbesondere in technischen Bereichen wie der Softwareentwicklung, praxisbezogene und kontextbezogene Fragen zu stellen. Diese Ansätze sollten auch in \acs{KI}-gestützte Rekrutierungsprozesse integriert werden, um eine umfassendere Bewertung der Kandidaten zu ermöglichen \cite{alami:2024}.

Die Frage der Fairness spielt bei der Anwendung von \acs{KI} in der Rekrutierung ebenfalls eine zentrale Rolle. Während \acs{KI}-Systeme Konsistenz in der Entscheidungsfindung bieten können, ist dies oft mit einem Kompromiss zwischen Fairness und Genauigkeit verbunden \cite{hunkenschroer:2022}. \acs{KI} bietet zwar das Potenzial, traditionelle Bewertungsmethoden zu verbessern, doch die Gefahr bleibt bestehen, dass bestehende Ungleichheiten verstärkt werden, wenn die Algorithmen nicht sorgfältig entwickelt und trainiert werden. Transparenz in der Funktionsweise von \acs{KI}-Systemen ist daher entscheidend, um sicherzustellen, dass Unternehmen und Bewerber nachvollziehen können, wie Entscheidungen getroffen werden.

\acs{KI}-gestützte Rekrutierungssysteme bieten viele Vorteile, insbesondere durch die Automatisierung und Skalierbarkeit von Prozessen. Sie helfen, datenbasierte Entscheidungen zu treffen und können menschliche Vorurteile reduzieren. Trotzdem gibt es noch einige Herausforderungen, vor allem bei der Bewertung von Soft Skills, der Vermeidung von Bias und der Gewährleistung von Fairness und Transparenz. Unternehmen, die \acs{KI} im Bewerbungsprozess nutzen, sollten sich dieser Probleme bewusst sein und Maßnahmen ergreifen, um Bias zu minimieren. Eine Möglichkeit besteht darin, die \acs{KI}-Systeme kontinuierlich zu überwachen und zu verbessern, um sicherzustellen, dass sie ethische und faire Entscheidungen treffen. Darüber hinaus bleibt der menschliche Faktor unerlässlich, um die Technologie zu überwachen und sicherzustellen, dass sie den Unternehmenszielen und ethischen Standards entspricht \cite{rigotti:2024}.
