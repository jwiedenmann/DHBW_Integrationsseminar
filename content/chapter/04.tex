%!TEX root = ../../main.tex

\chapter{Ergebnisse}

In diesem Kapitel werden die Ergebnisse der Bewerbervorauswahl vorgestellt. Ziel der Evaluation war es, herauszufinden, welches System besser geeignet ist, um geeignete Kandidaten bereits in der Vorauswahl zu identifizieren. Durch die Analyse der Interaktionen und Bewertungen von Testkandidaten durch beide Systeme sollen fundierte Rückschlüsse auf deren jeweilige Stärken und Schwächen gezogen werden.

\section{Überblick der Evaluation}

Im Rahmen des Experiments simulierte das LLM einen Chatbot, der auf Basis einer festen Stellenbeschreibung Fragen an die Bewerber stellte. Der zentrale Fokus lag darauf, zu ermitteln, ob das LLM in der Lage ist, Bewerber anhand ihrer Antworten in den Bereichen fachliche Kompetenz, Problemlösungsfähigkeit, Konfliktlösung und Teamfähigkeit objektiv zu bewerten. Dazu wurden die Antworten der Teilnehmer numerisch und qualitativ ausgewertet.

Zehn Personen nahmen an dem Experiment teil, wobei die Mehrheit aus der IT-Branche kam. Sieben der Teilnehmer hatten potenziell das Qualifikationsniveau für die bewertete Position, während drei Personen ohne Bezug zur Softwareentwicklung als Kontrollgruppe dienten. Ziel war es, die Fähigkeit des LLM zu testen, die Eignung von Kandidaten mit unterschiedlichem Erfahrungsgrad korrekt einzuschätzen.

Die vom LLM gestellten Fragen sowie die Bewertungen der Antworten wurden analysiert, um Rückschlüsse auf die Effektivität des LLM in der Vorauswahl von Bewerbern zu ziehen.

\section{Zusammenfassung der Ergebnisse}
