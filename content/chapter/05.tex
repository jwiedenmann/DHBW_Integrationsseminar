%!TEX root = ../../main.tex

\chapter{Schlussbetrachtung}

\section{Fazit}

Die vorliegende Arbeit untersuchte den Einsatz eines \acs{LLM}-gestützten Chatbots zur Optimierung der Bewerbervorauswahl im IT-Recruiting. Die Ergebnisse zeigen, dass der Chatbot in der Lage ist, eine erste Auswahl an Bewerbern anhand von festgelegten Bewertungskriterien durchzuführen. Dies bietet erhebliche Vorteile in Bezug auf die Effizienz und Objektivität des Auswahlprozesses. Der Einsatz von \acs{KI} zur Bewertung der fachlichen Kompetenz, Problemlösefähigkeit, Konfliktlösefähigkeit und Teamfähigkeit hat das Potenzial, die Personalrekrutierung in der IT-Branche zu revolutionieren. Dennoch bedarf es weiterer Feinabstimmungen und Verbesserungen des Systems, um die Bewertungsergebnisse zu präzisieren und Verzerrungen zu minimieren.

\section{Implikationen für Wissenschaft und Praxis}

Die in dieser Arbeit gewonnenen Erkenntnisse bieten wichtige Implikationen für sowohl die wissenschaftliche Forschung als auch die Praxis. In der Wissenschaft erweitert die Arbeit das Verständnis über den Einsatz von \acp{LLM} in personalwirtschaftlichen Prozessen und trägt zur Diskussion über die Automatisierung von Bewerberauswahlverfahren bei. Für die Praxis bietet der entwickelte Chatbot eine vielversprechende Möglichkeit, die Effizienz und Objektivität von Rekrutierungsprozessen zu steigern. Vor allem in der IT-Branche, wo spezifische Kompetenzen und schnelles Handeln gefragt sind, kann der Einsatz solcher Systeme zu einer erheblichen Zeit- und Ressourcenersparnis führen.

\section{Limitationen}

Trotz der positiven Ergebnisse weist das Projekt mehrere Limitationen auf. Zum einen bleibt die Fähigkeit des Chatbots, soziale und emotionale Kompetenzen der Bewerber präzise zu bewerten, begrenzt. Auch wenn technische Fähigkeiten gut erfasst werden, ist die Bewertung von Soft Skills eine Herausforderung, die weiter erforscht werden muss. Zudem besteht das Risiko, dass unbewusste Vorurteile in den Trainingsdaten des \acs{LLM}s die Objektivität des Auswahlprozesses beeinträchtigen.

Ein weiterer wesentlicher Punkt betrifft die geringe Anzahl an Teilnehmern im durchgeführten Experiment. Die Testgruppe war vergleichsweise klein, wodurch sich die Ergebnisse nur eingeschränkt auf eine größere Population übertragen lassen. Um signifikantere und allgemeinere Aussagen über die Effektivität des Systems treffen zu können, wären Experimente mit einer größeren und diverseren Stichprobe erforderlich. Des Weiteren basiert die Evaluierung der Bewerber auf statischen Kriterien, die möglicherweise nicht ausreichend flexibel auf unterschiedliche Bewerberprofile reagieren.

\section{Ausblick}

Der Einsatz von \acs{KI} in der Bewerbervorauswahl steht erst am Anfang und bietet vielversprechende Zukunftsperspektiven. Zukünftige Forschung sollte sich darauf konzentrieren, die Bewertung von Soft Skills durch den Chatbot weiter zu verfeinern und mögliche Bias-Probleme zu minimieren. Zudem könnte die Erweiterung der Datengrundlage sowie die Implementierung adaptiver Bewertungsmechanismen die Genauigkeit der Bewerberbeurteilung erhöhen. Ein weiterer spannender Forschungsansatz wäre die Integration des Chatbots in andere Bereiche des Personalmanagements, wie etwa die Karriereentwicklung oder das Performance-Management, um den gesamten Lebenszyklus der Mitarbeiter in einem Unternehmen besser zu unterstützen.